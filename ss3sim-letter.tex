\documentclass[letterpaper,11pt]{letter}
\usepackage{geometry}
\geometry{letterpaper}
\usepackage{graphicx}
\usepackage{fontspec}
\usepackage{pdfsync}
\defaultfontfeatures{Ligatures=TeX}
\setsansfont{Calibri}
\setmonofont{Inconsolata}
\setmainfont{Linux Libertine}
\topmargin -3cm
\textheight 27cm
\address{Sean C. Anderson, sean@seananderson.ca\\
  Biological Sciences, Simon Fraser University\\
  Burnaby, BC, V5A 1S6, Canada}

\begin{document}
\begin{letter}{}
\pagestyle{empty}
\opening{Dear Editor}

% You should supply an approximately one page cover letter that:
% Concisely summarizes why your paper is a valuable addition to the scientific literature
% Briefly relates your study to previously published work
% Specifies the type of article you are submitting (for example, research article, systematic review, meta-analysis, clinical trial)
% Suggests appropriate PLOS ONE Academic Editors to handle your manuscript (view a complete listing of our academic editors)
% Lists any recommended or opposed reviewers

Please find enclosed a manuscript entitled ``ss3sim: An R package for fisheries
stock assessment simulation with Stock Synthesis'' that we wish to submit for
possible publication as a research article in PLOS ONE.

Simulation testing is an important component to developing, verifying, and
understanding increasingly complex fishery stock assessment methods. One of the
more widely used stock assessment methods is Stock Synthesis 3 (SS3). However,
there lacks a generalized framework for simulation testing with SS3 --- most
work to date has used custom frameworks tailored to the particular needs of
each study.

In our paper, we introduce ss3sim, an R package that facilitates large-scale,
rapid, and reproducible simulation testing with Stock Synthesis. We describe
how ss3sim works, illustrate a simple example with code, compare the package to
other related software, and outline important research questions our package
could address. We also include an R vignette as Supplementary Material, which
describes in greater detail how to use the package and contains reproducible
example code. To our knowledge, this is the first generalized and open-source
software package focused specifically on stock assessment stimulation testing.

We confirm that this manuscript has not been published elsewhere and is not
under consideration by another journal. All authors have approved the
manuscript and agree with its submission to PLOS ONE.

We suggest the following possible Academic Editors:
Steven J. Bograd,
Daniel Duplisea,
Brian R. MacKenzie,
James P. Meador,
Jeffrey Buckel.

We suggest the following possible independent referees:

\begin{description}
  \item [Mark N. Maunder] mmaunder@iattc.org, Inter-American Tropical Tuna
    Commission
  \item [Kevin R. Piner] kevin.piner@noaa.gov, NOAA Fisheries, Southwest
    Fisheries Science Center
  \item [Arni Magnusson] arnima@hafro.is, Marine Research Institute, Reykjavik,
    Iceland
  \item [Alan C. Hicks] allan.hicks@noaa.gov, NOAA Fisheries, Northwest
    Fisheries Science Center
  %\item [Jonathan J. Deroba] jonathan.deroba@noaa.gov, National Marine Fisheries
    %Service, Northeast Fisheries Science Center
  \item [Simon D. Hoyle] simonh@spc.int, Oceanic Fisheries Program, Secretariat
    of the Pacific Community, New Caledonia
  \item [David B. Sampson] david.sampson@oregonstate.edu, Oregon State
    University
\end{description}

Thank you for considering our manuscript for publication and we look forward to
hearing from you.

\bigskip
\closing{Sincerely,\\
Sean Anderson, Cole Monnahan,\\Kelli Johnson, Kotaro Ono,\\Juan Valero}

\end{letter}
\end{document}
